\documentclass[10pt]{article}
\usepackage[margin=1in]{geometry}
\usepackage{hyperref}

\usepackage{titlesec}

\usepackage{tocloft}
\renewcommand{\cftsecnumwidth}{40pt}
\renewcommand{\cftsubsecnumwidth}{50pt}

\renewcommand*{\thesection}{\Roman{section}}

\titleformat{\section}{\Large \bfseries}{Bylaw \thesection.}{1em}{}

\titleformat{\subsection}{\bfseries}{Section \thesubsection.}{1em}{} 

\usepackage{xcolor}
\newcommand{\red}[1]{{\color{red} #1}}
\newcommand{\blue}[1]{{\color{blue} #1}}

\newcommand{\blref}[1]{Bylaw \ref{#1}}
\newcommand{\secref}[1]{Section \ref{#1}}

\begin{document}

\tableofcontents

%%%%%%%%%%%%%%%%%%%%%%%%%%%%%%%%%%%%%%%%%%%%%%%%%%%%%%%%%%
\section{Rules of Order}
\subsection{Robert's Rules of Order}

In all matters not covered by the Constitution, Bylaws, and Policies and Procedures, the Administrative Committee (AdCom) shall be governed by the latest edition of Robert's Rules of Order.

\section{Elections}
\subsection{Nominations Committee}
\label{Elections:NominationsCommittee}

The Nominations Committee is appointed by the President in accordance with \blref{StandingCommittees}. Within one month of its formation but no later than March 1, the Nominations Committee shall develop and present to AdCom its schedule of elections for the calendar year such that 1) results of the AdCom election scheduled for the year will be announced no later than December 15, and 2) the slate of Society officer candidates will be announced to AdCom at least 30 days prior to officer elections (for President) or confirmations (for Vice Presidents). This schedule must receive AdCom approval within one month of presentation but no later than June 30. The Nominations Committee shall be chaired by the Junior Past-President.  In the event of the incapacity or conflict of interest of the Chair, the most recent Past Chair of the Nominations Committee available shall be the Chair of the Nominations Committee.  Under extenuating circumstances, a different individual may be appointed to this position.  Chairs shall not be eligible to be elected to the AdCom during their term of service.

A member of the Nominations Committee may be nominated and run for a position for which the Nominations Committee is responsible for making nominations only on the following conditions: 1) the nomination is not made by a member of the same Nominations Committee and 2) the member resigns from the Nominations Committee prior to its first meeting of the year in which the nomination shall be made.


\subsection{Notice to Members}

The Secretary shall notify all members of the Society, through the Transactions/Magazine of the Society and by electronic newsletter, that individual voting members eligible to vote in the election of Administrative Committee members may nominate candidates for the Administrative Committee by a written petition or by a majority vote at a meeting (in person or electronic) of the Nominations Committee, provided such nominations are received in accordance with the schedule set by the Nominations Committee, and are received at least 120 days before the date of election. 


\subsection{Nominations}
In accordance with the announced schedule of elections, the Nominations Committee shall submit to the Secretary of the Society a slate of candidates, according to the Society Policies and Procedures, to replace the AdCom members whose terms expire December 31 of that year. A slate of nominees for member-at-large comprising a minimum of 1.5 times the number of vacancies to occur on the AdCom shall be put forth by the Nominations Committee. In the event that the minimum number of nominations is not received, the IEEE Vice President for Technical Activities may allow a smaller slate. (IEEE TAB Operations Manual 4.2.B.5) In accordance with the IEEE Bylaws the duties of the Nominations Committee shall also include the annual solicitation of names of potential candidates to be considered by the IEEE Nominations and Appointments Committee for service on Institute Committees and Boards. The Nominating Committee for the IEEE Division X Director shall be a separate Committee and shall operate as specified in the TAB Operations Manual.

In the case of candidates nominated by petition, the number of signatures on a petition shall be at least 2\% of the eligible voters according to the official membership counted at the end of the year preceding the election. Members shall be notified of all duly made nominations prior to the election.  Prior to the submission of a nomination petition, the petitioner shall have determined that the nominee named in the petition is willing to serve, if elected; evidence of such willingness to serve shall be submitted with the petition. 

Signatures may be submitted electronically through an official IEEE petition website set up for a nominee, or by signing and mailing a paper petition.  The name of each member signing the paper petition shall be clearly printed or typed. For identification purposes of signatures on paper petitions, membership numbers or addresses as listed in the official IEEE membership records shall be included.  Only signatures submitted electronically through the IEEE petition website or original signatures on paper petitions shall be accepted.  Facsimiles, or other copies of the original signature, shall not be accepted.



\subsection{Administrative Committee Election}

In accordance with the announced schedule of elections, the Secretary shall forward to IEEE headquarters the information for the ballot containing all the nominees for election to the AdCom, including those submitted by the Nominations Committee and those properly nominated by petition.  IEEE headquarters shall handle the conduct of the election, providing all members of the Society the opportunity to vote.  It shall count the returned ballots with the understanding that the newly elected AdCom members will be decided before January 1 of the year following that in which the election was initiated.

\subsection{Election of the President-Elect}

In the year prior to the current President-Elect assuming the Presidency, and in accordance with the announced schedule of elections, the AdCom shall hold a meeting to elect the next President-Elect.  The Nominations Committee is responsible for ensuring that there is at least one nominee for this position, in accordance with the Society Policies and Procedures. Other candidates may be nominated, also in accordance with the Society Policies and Procedures. A majority of those voting is required for election.  If no candidate receives a majority vote, the individual with the least number of votes shall be dropped and a new vote taken. In the case of a tie for the least number of votes, a runoff between those candidates will be held. 

The name of the elected President shall be reported to the Chair and to the Secretary of the IEEE Technical Activities Board.

\subsection{ Confirmation of the Vice Presidents-Elect}

No later than June 30, during the second year of service of the incumbent President-Elect, and in accordance with the schedule of elections in effect, the AdCom shall hold a meeting for confirmation votes on all the Vice Presidents-Elect who will serve during the President-Elect's term as President. The Nominations Committee will support the President-Elect in selecting the slate of candidates for the Vice Presidents-Elect, and shall submit to the AdCom the nominee for each of these offices in accordance with the Society Policies and Procedures. No Vice President shall serve for more than two consecutive two-year terms. For at least one of the Vice President-Elect offices, the nominee shall be a Society member who has never served before on the Society Executive Committee. 

The names of the elected officers shall be reported to the Chair and to the Secretary of the IEEE Technical Activities Board.

\end{enumerate}

\subsection{Principles of Selection}

The Nominations Committee, in its nominations, and the AdCom, in its elections, shall be guided in their selections of candidates by the principles of demonstrated or potential ability and by other factors which may be relevant, such as geographical distribution (\secref{AdCom:Composition}), gender diversity, career level, and the academic, governmental, and industrial distribution of the candidates.  


\subsection{President-Elect Appointments}

Prior to becoming President, the President-Elect shall submit to the incumbent AdCom the names of a proposed Secretary, Parliamentarian and Treasurer, plus, if not otherwise specified in these Bylaws, names for all the Standing Committee Chairs, all the Technical Committee Chairs, other Committee Chairs, and Liaison Representatives for appointment. The President-Elect shall be guided in selections of candidates by the principles of demonstrated or potential ability and by other factors which may be relevant, such as geographical distribution (\secref{AdCom:Composition}), gender diversity, career level and the academic, governmental, and industrial distribution of the candidates.    If a majority of the members of said AdCom do not object within thirty (30) days from date of formal submission, appointments shall become final.  If a majority of the members of said AdCom object, a new name (or names) must be submitted. The incumbent officials shall remain in office until successors are appointed and arranged to take over the offices.

%%%%%%%%%%%%%%%%%%%%%%%%%%%%%%%%%%%%%%%%%%%%%%%%%%%%%%%%%%%%%%%%%%
\section{Duties of Executive Officers}

\subsection{Secretary}

The Secretary shall function as the Secretary of the Society and shall also be responsible for non-membership administrative aspects of the Society. These non-membership administrative responsibilities include serving as a communications channel for the following Standing Committees: (1) Standards Committee, (2) the Constitution and Bylaws Committee, and (3) Nominations Committee. In particular, the Secretary shall have the responsibility of seeing that the Society operates in accordance with the rules of the Society Constitution and Bylaws.  Secretarial duties shall include the responsibility for sending out notices according to plans delineated by the AdCom or laid down in the Society Bylaws, and preparing the agenda for and recording the minutes of all meetings of the AdCom and general meetings of the Society and making such reports of these activities as may be required by the AdCom, the IEEE Technical Activities Board or the IEEE Bylaws.


\subsection{Treasurer}

The Treasurer shall handle all expenses of the Society, including approval of travel expense reports and Society meeting expenses. The Treasurer shall act as consultant and financial supervisor to the treasurers of the various meetings held by the Society, and make periodic reports to the AdCom. The Treasurer shall also monitor all Society income, e.g., from membership, conferences, publications, and, in case of irregularity, promptly inform the Vice President for Financial Activities and the President. The Treasurer shall serve on the Financial Activities Board and work in strict cooperation with the Vice President for Financial Activities to help the Society to stay on budget.  


\subsection{Parliamentarian}

The Parliamentarian is a consultant officer who has the responsibility to advise the President and other Officers, Boards, Committees and members on matters of parliamentary procedure. The Parliamentarian is appointed by the President with the concurrence of the AdCom, and he/she shall serve as an ex officio member of the AdCom without vote.


\subsection{Vice President for Financial Activities}

The Vice President for Financial Activities shall have overall responsibility for the finances of the Society, including the general responsibility for procuring more funds for the Society. Up to four Associate Vice Presidents may be appointed as needed, by the Vice President for Financial Activities in consultation with the President. The Vice President for Financial Activities shall chair a Financial Activities Board (\blref{FAB}). The Vice President for Financial Activities, with the assistance of the Financial Activities Board, shall prepare the Society budgets which shall then be submitted to the AdCom for approval at the Society Annual Meeting.


\subsection{Vice President for Member Activities}
The Vice President for Member Activities shall be responsible for all membership-related activities of the Society, and in particular for member services, for ensuring the growth of the membership of the Society and for paying special attention to the membership of the Society worldwide. Up to four Associate Vice Presidents may be appointed as needed, by the Vice President for Member Activities in consultation with the President. The Vice President for Member Activities shall be responsible for the following Standing Committees: the Chapter and International Activities Committee, the Human Rights and Ethics Committee, the Membership, Admissions and Retention Committee, the Member Services Committee, the Student Activities Committee, the Women in Engineering Committee, the Young Professionals Committee, the Life Member Committee, and the Special Interest Group on Humanitarian Technology. The Vice President for Member Activities, with consultation of the RAS President shall appoint the respective Committee Chairs to IEEE Young Professionals Committee, IEEE Women in Engineering Committee, and the IEEE Life Member Committee, and the Special Interest Group on Humanitarian Technology.

The Vice President for Member Activities, with consultation of the RAS President shall appoint the respective Committee Chairs to IEEE Young Professionals Committee, IEEE Women In Engineering Committee, and IEEE Life Member Committee.

The Vice President for Member Activities shall chair a Member Activities Board (\blref{MAB}). 



\subsection{Vice President for Technical Activities}
The Vice President for Technical Activities shall be responsible for all the technical activities of the Society, including the technical contents of Society meetings. In particular, the Vice President for Technical Activities shall have direct responsibility for the following: The Technical Committees and the Working Groups. Up to four Associate Vice Presidents may be appointed as needed, by the Vice President for Technical Activities in consultation with the President.

The Vice President for Technical Activities shall chair a Technical Activities Board (\blref{TAB}). 



\subsection{Vice President for Publication Activities}

The Vice President for Publication Activities shall have overall responsibility for all publications of the Society, including Conference Proceedings whose responsibility will be shared with the Vice President for Conference Activities, whether in printed or electronic form.  Up to four Associate Vice Presidents may be appointed as needed, by the Vice President for Publication Activities in consultation with the President. 

The Vice President for Publication Activities shall chair a Publications Activities Board (\blref{PAB}). 



\subsection{Vice President for Conference Activities}

The Vice President for Conference Activities has overall responsibility for all Conferences, Symposia, Meetings, Workshops and Events to which the Society lends its name. This includes the annual International Conference on Robotics and Automation, 100\% sponsored meetings, partial and co-sponsored meetings, technically co-sponsored meetings, and competitions that use the Society's name. 

The Vice President for Conference Activities shall chair the Conference Activities Board (\blref{CAB}) and shall negotiate with the IEEE and sister societies on all matters relating to jointly sponsored conferences and meetings.  

The Associate Vice President – Conference Finance (CAB-F), the Associate Vice President – Technical Program (CAB-T), the Associate Vice President – Conference Operations (CAB-O), the Associate Vice President – Conference Publications (CAB-P), the Associate Vice President for Conference Competitions, the Chair of the Publication Ethics Committee shall report to the Vice President for Conference Activities.



\subsection{Vice President for Industrial Activities}

The Vice President for Industrial Activities shall be responsible for all industrial related matters and the growth of the industrial community within RAS, both by promoting the participation of industrial partners in current activities and thus creating more links with academics, and by proposing new actions beneficial to this community. Up to four Associate Vice Presidents may be appointed as needed, by the Vice President for Industrial Activities in consultation with the President. 

The Vice President for Industrial Activities shall chair the Industrial Activities Board (\blref{IAB}). 



\subsection{Vice President for Media Services}

The Vice President for Media Services (previously called the Vice President for Electronic Products and Services) shall be responsible for all matters related to electronic products and services. Up to four Associate Vice Presidents may be appointed as needed, by the Vice President for Media Services, in consultation with the President. 

The Vice President for Media Services shall chair the Media Services Board (\blref{MSB}).



\subsection{Transactions Editors-in-Chief}

The Transactions Editors-in-Chief shall be charged with efficient operation of the Society Transactions.  Papers for the Transactions shall be received by the Editors-in-Chief of the Society Transactions whose office shall serve as a focal point for processing these papers. The Transactions Editors-in-Chief have the responsibility for recommending to the Vice President for Publication Activities the number of Editors and Associate Editors of the Society Transactions to be appointed, the technical areas that each of the Associate Editors shall cover, and suitable candidates for appointment as Editors and Associate Editors. Terms of Office are described in \secref{TEB:Composition}.


\subsection{Letters Editors-in-Chief}

The Letters Editors-in-Chief shall be charged with efficient operation of the Society Letters.  Papers for the Letters shall be received by the Editors-in-Chief of the Society Letters whose office shall serve as a focal point for processing these papers.  The Letters Editors-in-Chief has the responsibility for recommending to the President the Deputy Editor-in-Chief, the number of Editors and Associate Editors of the Society Letters to be appointed, the Technical areas that each of the Associate Editors shall cover, and suitable candidates for appointment as Editors and Associate Editors. Terms of Office are described in \secref{LEB:Composition}.


\subsection{Magazine Editor-in-Chief}

The Magazine Editor-in-Chief shall be charged with efficient operation of the publication of the Society Magazine. Members can submit information that is of interest to the Society to the Magazine Editor-in-Chief. The office of the Magazine shall serve as a focal point for processing the information. The Magazine Editor-in-Chief has the responsibility for recommending to the President the number of Editors and Associate Editors of the Society Magazine to be appointed, the technical areas that each of the Associate Editors shall cover, and suitable candidates for appointment as Editors and Associate Editors. Terms of Office are described in \secref{MEB:Composition}. 


\subsection{Vice Presidents-Elect}

The Vice Presidents-Elect shall serve as ex officio members without vote on the Society Board chaired by the respective incumbent Vice President.


\section{Administrative Committee (AdCom)}
\label{AdCom}

\subsection{Composition}
\label{AdCom:Composition}

The AdCom is chaired by the President and it is composed of the following voting members: the President, the Junior Past-President, the President-Elect, the Secretary, the Treasurer, the Chair of the Student Activities Standing Committee and the eighteen elected members.  Additional ex officio AdCom members without vote are as prescribed in these Bylaws. The AdCom composition of elected members shall reflect the geographical distribution of the membership worldwide in accordance with the Society Policies and Procedures.


\subsection{Annual Meetings}
The AdCom shall hold an Annual Meeting before June 30.


\subsection{Meeting}

No meeting of the AdCom shall be held for the purpose of transacting business unless each member shall have been sent notice of the time and place of such meeting twenty (20) days prior to the date scheduled for the meeting. 


\section{Financial Activities Board}
\label{FAB}

\subsection{Purpose}
This Board shall act as the liaison for the Society on financial matters with IEEE Headquarters; act as consultant and financial supervisor of the RAS Treasurer; make periodic reports to the AdCom on the financial status of the Society; and observe the financial operations of the Society and take appropriate actions to see that money is spent or invested wisely and in the best interest of the Society.


\subsection{Composition}
This Board shall consist of the Chair, the President, the Vice President for Publication Activities, the Vice President for Conference Activities, the Secretary, the Treasurer, the President Elect (the Chair of the Long Range Planning Committee), the Vice President for Member Activities, the Vice President for Technical Activities, the Vice President for Industrial Activities, and the Vice President for Media Services, plus between one and five additional members appointed by the Vice President for Financial Activities with approval of the President. At least one member of the board must be an elected member of the AdCom. At least one member of the board must be the past Vice President for Financial Activities.


\subsection{Actions}
The Financial Activities Board shall meet at least two times a year to review the financial activities of the Society. 


\section{Member Activities Board}
\label{MAB}
\subsection{Purpose}
The Member Activities Board shall be responsible for all membership-related activities of the Society, and in particular for member services, for ensuring the growth of the membership of the Society and for paying special attention to the membership of the Society worldwide.

\subsection{Composition}
This Board shall consist of the Chair (Vice President for Member Activities), the President, all Associate Vice Presidents of Member Activities, Chairs, and co-chairs of the Standing Committees for which the Board is responsible, plus between one and five additional members appointed by the Vice President for Member Activities with approval of the President. At least one member of the board must be an elected member of the AdCom. This Board has the responsibility for overseeing the activities of the Standing Committees for which the Board is responsible and for any other membership-related activities.


\subsection{Actions}
The Member Activities Board shall meet at least two times a year to review the member activities of the Society.


\subsection{Chapter and International Activities Committee}

This committee shall report to the Member Activities Board, and is in charge of organizing and fostering international cooperation among scholars and students. An important means towards this goal are Section and Student Branch Chapters of the Society, which will report to this Standing Committee.  A concern of the Standing Committee shall be to ensure that members and students of all regions are well represented and feel equally at home in the diverse Society activities (conferences, schools, ballots, etc.).


\subsection{Human Rights and Ethics Committee}

This committee shall report to the Member Activities Board. The Committee shall be responsible for making recommendations to the AdCom for actions to be taken on behalf of the Society for the benefit of members or colleagues whose professional activities have been severely hampered by governmental, institutional, or other authorities.    It shall also be responsible for making recommendations on issues such as conflict of interest, student/teacher relations, etc., and promulgating information on these matters to the membership. 


\subsection{Membership, Admissions and Retention Committee}

This committee shall report to the Member Activities Board. The Committee shall be responsible for encouraging membership in the Society by all members of the IEEE who are interested in the Field of Interest of the Society, and by non-IEEE members as Affiliate Members. It shall also focus on membership retention.


\subsection{Member Services Committee}
This committee shall report to the Member Activities Board. The objective of this committee is to improve the services provided to the Society’s members. The emphasis will be on the mentoring and guidance of junior members of the Society along with a variety of new services (e.g., professional and networking) to the more senior members.  


\subsection{Student Activities Committee}

This committee shall report to the Member Activities Board. The Committee shall promote student participation in the Society activities in cooperation with all Society Boards, Committees and Working Groups. The Chair of this Committee shall be appointed by the President from a slate of candidates Society student members recommended by the Member Activities Board to serve for a two-year term. The Chair of the Student Activities Standing Committee shall be an ex officio AdCom member with vote. The co-chairs of this committee shall be appointed by the President from a slate of candidate Society student members recommended by the Member Activities Board to serve a one-year term, renewable once.


\subsection{Women in Engineering Committee}

This committee shall report to the Member Activities Board. The Committee shall promote Women in Engineering through Society activities in cooperation with all Society Boards, Committees and Working Groups. The Chair of this Committee or one of its Co-Chairs shall serve as RAS representative to the IEEE Women in Engineering Committee.


\subsection{Young Professionals Committee}

This committee shall report to the Member Activities Board. The Committee shall promote Young Professional participation in the Society activities in cooperation with all Society Boards, Committees and Working Groups. The Chair of this Committee or one of its Co-Chairs shall serve as RAS representative to the IEEE Young Professionals Committee.


\subsection{Life Member Committee}

This committee shall report to the Member Activities Board. The committee shall promote the activities of Life Members in IEEE RAS through Society activities in cooperation with all Society Boards, Committees and Working Groups. It shall also support activities of the local chapters and society committees as mentors and will serve in an advisory role for RAS activities. The Chair of this Committee shall serve as RAS representative to the IEEE Life Member Committee.


\subsection{Special Interest Group on Humanitarian Technology (SIGHT)}

This committee shall report to the Member Activities Board. The mission of RAS-SIGHT is the application of robotics and automation technologies for promoting humanitarian causes around the globe, and to leverage existing and emerging technologies for the benefit of humanity and towards increasing the quality of life in underserved, underdeveloped areas in collaboration with existing global communities and organizations.


\section{Technical Activities Board}
\label{TAB}

\subsection{Purpose}

This Board shall be responsible for acquainting the members of the Society, in particular, and the engineering community and the public, in general, with the state of the art and its progress within the Field of Interest of the Society by means of published and oral communications. It shall be responsible for coordinating all of the technical activities of the Society.


\subsection{Composition}

This Board shall consist of the Chair (Vice President for Technical Activities), the President, all Associate Vice Presidents Technical Activities, the Editors-in-Chief of the Transactions, Letters and Magazine, the Vice President for Conference Activities, the Chairs of the Technical Committees, the Vice President for Media Services, plus between one and five additional members appointed by the Vice President for Technical Activities with the approval of the President. At least one member of the board must be an elected member of the AdCom. This Board shall be responsible for coordinating all the technical activities of the Society.


\subsection{Actions}

The Technical Activities Board shall meet at least two times a year to review the technical activities of the Society. 


\section{Technical Committees and Working Groups}
\label{TCs}



\subsection{Purpose}
Technical Committees are established by the Technical Activities Board and approved by the AdCom to provide a focus for the technical activities of the Society independent of the Conference Activities Board and of the Editorial Boards of the Transactions. Each Technical Committee may sponsor and monitor a number of Working Groups.



\subsection{Working Groups}

Nonregional subgroups of members of the Society who share common technical interests and needs may be formed by petition to the AdCom of the Society. Each Working Group must be formed and operated under a plan consistent with the Society Constitution and Bylaws and not inconsistent with the authority delegated to the AdCom. The Chairs of each Working Group shall report directly to the Chair of the Technical Committee that sponsors and monitors that Working Group.  Upon the approval of the relevant Society Boards, these groups may hold special workshops or sessions in their technical areas during Society meetings. They may request from the Editorial Boards of the Transactions to have special issues published in the Transactions. The actual interfacing with these Boards must be done in conjunction with the Technical Committee Chair responsible for the particular Working Group in question. The number and types of Working Groups associated with a given Technical Committee may change depending upon the current interests of the membership of the Society.  Such groups may be formed or dissolved upon the approval of the AdCom.


\subsection{Technical Committee Chairs}

Technical Committee Chairs proposed by the Vice President for Technical Activities shall be approved by the President. No Technical Committee Chair shall serve for more than two consecutive three-year terms, with an exception of one year transition period for a maximum of two co-chairs. Eligibility is restored after a lapse of three years. These Chairs shall be charged with the guidance of the regular activities of the Technical Committees. The VP for Technical Activities shall be responsible for making yearly reports to the AdCom on the activities of the Technical Committees and the Working Groups monitored by the Technical Activities Board. These reports must include recommendations on the continuation or dissolution of these Working Groups and Committees for the next year.


\section{Publication Activities Board}
\label{PAB}

\subsection{Purpose}
The Publication Activities Board shall have overall responsibility for all publications of the Society whether in printed or electronic form, excluding Conference Proceedings.  




\subsection{Composition}

The Publication Activities Board consists of voting members and a consultative body. 

The voting members are composed of the PAB Executive Committee (\secref{PAB:ExCom}), the Editors-in-Chief (EICs) of the Transactions, Letters and Magazine, the Vice President for Conference Activities, the Chair of the Publication Ethics Committee and one representative from each co-sponsored journal in which the Society participates. 

The consultative body is composed of one representative from each Society Transactions and Letters appointed by the respective EIC, the Immediate Past Transactions and Letters EICs, one representative from the Long Range Planning Committee, one representative from the Technical Activities Board, the Vice President of Media Services and between one and five additional  members appointed by the Vice President for Publication Activities with approval of the President. At least one member of the board must be an elected member of the AdCom.


\subsection{Actions}

The Publication Activities Board shall meet at least two times a year to review the technical activities of the Society.

\subsection{PAB Executive Committee}
\label{PAB:ExCom}

The PAB Executive Committee will consist of the Vice President for Publication Activities, the President, the Committees Chairs and EICs of fully sponsored RAS journals. The Vice President for Publication Activities shall chair the PAB Executive Committee.


\subsection{Co-sponsored Journals, New Journals and Other Publications Committee}

This committee shall report to the Vice President for Publication Activities. It is chaired by the Associate Vice President for co-sponsored journals, new journals and other publications and it is composed of up to three PAB members. The Committee will monitor performances of co-sponsored journals, will advise on selection of RAS representatives in co-sponsored journals and will evaluate launch of new journals. 

The Vice President for Publication Activities, the President of RAS and IEEE RAS Executive Director are ex-officio members.


\subsection{Periodical Promotion, New Channels and Finances Committee}

This committee shall report to the Vice President for Publication Activities. It is chaired by the Associate Vice President for periodical promotion, new channels and finances and it is composed of up to three PAB members. The Committee will investigate the best strategies for RAS sponsored journals promotion and will guarantee correct financial management.

The Vice President for Publication Activities, the President of RAS and IEEE RAS Executive Director are ex-officio members.


\subsection{Publication Quality Committee}

This committee is chaired by the Vice President for Publication Activities or one of the Associate VPs. It is composed of the EICs. The Committee will monitor the quality of the review process and of the publications of RAS sponsored journals by analyzing journal submission data in both a qualitative and quantitative manner and by implementing homogeneous performance metrics.

The President of RAS and IEEE RAS Executive Director are ex-officio members.



\subsection{Publication Operations Committee}

This committee shall report to the Vice President for Publication Activities. It is chaired by one RAS publication staff, or a PAB member and it is composed by the RAS publication staff (i.e. Editorial Assistants of RAS sponsored journals). The Committee will guarantee the maintenance of RAS sponsored journal high standards by monitoring timeliness and efficiency of peer-review, as well as the quality of administrative support to the individual journals.

The Vice President for Publication Activities, the President of RAS and IEEE RAS Executive Director are ex-officio members.


\subsection{Publication Ethics Committee}

This committee shall report to the Vice President for Publication Activities and Vice President for Conferences Activities. The Committee shall be responsible for making recommendations about plagiarism issues in RAS sponsored publications and conference proceedings.


\section{Transactions Editorial Boards}
\label{TEB}

\subsection{Purpose}

The Transactions publish high-quality papers on the theory, design, and applications of Robotics and Automation and areas as stated in the Field of Interest of the Society.

\subsection{Composition}
\label{TEB:Composition}
Each Editorial Board shall consist of the Editor-in-Chief (EIC), a number of Editors, a number of Associate Editors as regular members, and the Vice President for Publications and Magazine Editor-in-chief as ex officio members.  The normal terms of the Editors-in-Chief and Editors shall be five years, non-renewable, and the normal term of the Associate Editors shall be one year, followed by an additional two to three years if the Editors and EICs evaluate their performance sufficiently positively. The Editors-in-Chief shall be appointed by the Society President, with the approval of RAS AdCom, one year in advance of the expiration of the term of the current EICs. The Editors and Associate Editors are appointed from time to time as needed by the VP for Publication Activities upon recommendation from the Editors-in-Chief.

\subsection{Actions}

Each Editorial Board of the Transactions shall be chaired by the Editor-in-Chief of the Transactions and shall meet at least two times a year to decide upon policy issues for publication in the Transactions, and management issues of importance to Editors and Associate Editors.  The Editorial Boards shall make decisions on the dispositions of papers throughout the year based upon the editorial reviews obtained by the members of the Editorial Boards, in accordance with IEEE policies and guidelines on Transactions publications. The Vice President for Publication Activities shall be responsible for making annual reports to the AdCom on editorial activities and plans for the coming year. 


\section{Letters Editorial Board}
\label{LEB}

\subsection{Purpose}
The Letters publishes high-quality papers on the theory, design, and applications of Robotics and Automation and areas as stated in the Field of Interest of the Society.


\subsection{Composition}
\label{LEB:Composition}

The Editorial Board shall consist of the Editor-in-Chief (EIC), a Deputy Editor-in-Chief, a number of Editors, a number of Associate Editors as regular members, and the Vice President for Publications, Conference Editorial Board Editor-in-Chief, and the Vice President for Conference Activities as ex officio members.  The normal terms of the Editor-in-Chief, Deputy Editor in Chief and Editors shall be three years, renewable once and the normal term of the Associate Editors shall be one year, followed by an additional two years if the Editors and EIC and Deputy EIC evaluate their performance sufficiently positively. Normally, Associate Editors shall not serve two consecutive terms without a gap of approximately one year.  The Editor-in-Chief and Deputy Editor-in-Chief shall be appointed by the Society President, with approval of RAS AdCom one year in advance of the expiration of the term of the current EICs. The Editors and Associate Editors are appointed from time to time as needed by the Vice President for Publication Activities upon recommendation from the Editor-in-Chief.


\subsection{Actions}

The Editorial Board of the Letters shall be chaired by the Editor-in-Chief of the Letters and shall meet at least two times a year to decide upon policy issues for publication in the Letters, and management issues of importance to the Deputy Editor-in-Chief, Editors and Associate Editors.  The Editorial Board shall make decisions on the dispositions of papers throughout the year based upon the editorial reviews obtained by the members of the Editorial Board, upon consideration of available space in the Letters and upon available monies for publication of the Letters.  Accepted papers need not be presented at any IEEE Meetings. The EIC, Deputy EIC and Editorial Boards shall be responsible for making annual reports to the AdCom on editorial activities and plans for the coming year.


\section{Magazine Editorial Board}
\subsection{Purpose}

The Magazine publishes high-quality articles on the theory, design, and applications of Robotics and Automation and areas as stated in the Field of Interest of the Society.


\subsection{Composition}
\label{MEB:Composition}
The Editorial Board shall consist of the Editor-in-Chief (EIC), a number of Editors as regular members, and the Vice President for Publications as ex officio member.  The normal term of the Editor-in-Chief be five years and the normal term of the Editors shall be one year, followed by an additional two years if the EIC evaluates their performance sufficiently positively. Normally, Editors shall not serve two consecutive terms without a gap of approximately one year.  The Editor-in-Chief shall be appointed by the Society President, with approval of RAS AdCom one year in advance of the expiration of the term of the current EIC.  The Editors are appointed from time to time as needed by the Vice President for Publication Activities upon recommendation from the Editor-in-Chief.


\subsection{Actions}
The Editorial Board of the Magazine shall be chaired by the Editor-in-Chief of the Magazine and shall meet at least two times a year to decide upon policy issues for publication in the Magazine, and management issues of importance to Editors.  The Editorial Boards shall make decisions on the dispositions of papers throughout the year based upon the editorial reviews obtained by the members of the Editorial Boards, upon consideration of available space in the Magazine and upon available monies for publication of the Magazine. Accepted papers need not be presented at any IEEE Meetings.


\section{Conference Activities Board}
\label{CAB}

\subsection{Purpose}

The Conference Activities Board (CAB) shall be responsible for the management, planning, and oversight of all conferences, symposia, workshops, and events (collectively referred to as conference activities) of the Society including all conferences, symposia, and workshops that carry the RAS brand name.  This responsibility extends to the publications associated with any of these conference activities. It shall also monitor and coordinate all conference operating committees and carry out all the long-term conference planning for all RAS conference activities. This Board shall be charged with keeping itself informed of all conferences, symposia, and workshops which are in areas covered by the Field of Interest of the Society and shall advise the AdCom about the participation of the Society in such meetings. In conjunction with the Technical Activities Board, this Board shall promote the participation of the Society in emerging technical areas through strategic planning and initiating new conference activities in these areas. In conjunction with the Member Activities Board, CAB will provide oversight on the selection and operation of the RAS Technical Education Programs and RAS sponsored Competitions. 


\subsection{Composition}

The Conference Activities Board consists of the CAB Executive Committee (\secref{CAB:ExCom}) and the Vice President for Financial Activities, the Vice President for Technical Activities, the Vice President for Publication Activities,  the Vice President for Industrial Activities, the Vice President for Media Services, the Treasurer, the SAC Chair or co-Chair, the Chairs or Co-chairs of the Steering Committees of all fully sponsored conferences, the Chair or Co-chair of the Steering Committee of the IEEE/RSJ International Conference on Intelligent Robots and Systems, and between one and five additional members appointed by the Vice President for Conference Activities with approval of the President. At least one member of the board must be an elected member of the AdCom.  

\subsection{Actions}

The Conference Activities Board shall meet at least two times a year to review the conference, symposia, workshop, and event activities of the Society.


\subsection{CAB Executive Committee}
\label{CAB:ExCom}

The CAB Executive Committee will consist of the Vice President for Conference Activities, the President, the Associate Vice President for Technical Programs, the Associate Vice President for Conference Finances, the Associate Vice President for Conference Publications, the Associate Vice President for Conference Operations, and the Associate Vice President for Conference Competitions. The Vice President for Conference Activities shall chair the CAB Executive Committee. The RAS Executive Director may name staff to serve as ex officio members of the committee in consultation with IEEE Meetings, Conferences and Events (MCE).


\subsection{Conference Finance Committee}

This committee shall report to the Vice President for Conferences Activities. The Conference Finance Committee (CAB-F) will consist of the Associate Vice President for CAB Finance, four past, current, or future Finance Chairs of fully sponsored conferences, and a member of the AdCom (appointed by the President of RAS).   The Vice President for Conference Activities, the President of RAS and the IEEE RAS Executive Director are ex-officio members. The Associate Vice President for Finance will chair the CAB Finance Committee. The Vice President for Conference Activities may appoint additional members to ensure that all financial aspects of conference management are represented. The Conference Finance Committee monitors the financial operations of the ongoing conferences. The Conference Finance Committee is in charge of acquiring the appropriate documentation for each conference and to start the CAB approval process. At the annual meetings, the Associate Vice President for Conference Finances presents the list of conferences requesting sponsorship and illustrates the documentation presented.

\subsection{Conference Operations Committee}

This committee shall report to the Vice President for Conferences Activities. The Conference Operations Committee (CAB-O) will consist of the Associate Vice President for CAB Operations, four past, current, or future members of conference organization teams of fully sponsored conferences, one member for workshop oversight, and a member of the AdCcom (appointed by the President of RAS).  The Vice President for Conference Activities, the President of RAS and the IEEE RAS Executive Director are ex-officio members. The Associate Vice President for Operations will chair the CAB Operations Committee. The Vice President for Conference Activities may appoint additional members to ensure and maintain high standards for conference organization, advice conference organizers, and coordinate activities between (i) conference organization teams, (ii) RAS Administrative Committee (AdCom), and (iii) RAS Executive Committee (ExCom). The committee maintains conference records encompassing the event organization and results.
The board member who oversees workshops is charged with maintaining year-to-year consistency in the quality of the workshops and ensuring they reflect the breadth of technical activities in RAS.



\subsection{Conference Publications Committee}

This committee shall report to the Vice President for Conferences Activities. The Conference Publications Committee (CAB-P) will consist of the Associate Vice President for CAB Publications, four past, current, or future Publications Chairs of fully sponsored conferences, and a member of the AdCom (appointed by the President of RAS). The Vice President for Conference Activities, the President of RAS and the IEEE RAS Executive Director are ex-officio members. The Associate Vice President for CAB Publications will chair the CAB Publications Committee. The Vice President for Conference Activities may appoint additional members to ensure that all aspects of conference publications are represented. The Conference Publication Committee shall be responsible for year-to-year consistency and coordination of the publication aspects of conferences such as: conference proceedings; conference digest; multimedia contents; tutorial/workshop proceedings; and other matters pertaining to the publication aspects of our conferences. The Conference Publications Committee ensures that conference proceedings are delivered to IEEE Xplore in a timely manner.



\subsection{Conference Technical Program Committee}
This committee shall report to the Vice President for Conferences Activities. The Conference Technical Programs Committee (CAB-T) will consist of the Associate Vice President for Technical Programs, four past, current, or future Program Chairs of fully sponsored conferences, the Editor-in-Chief for the ICRA Conference Editorial Board (see Article VIII), the Editor-in-Chief of the IROS Conference Paper Review Board, and one member each from the Publication Activities Board (appointed by the Vice President for Publication Activities), the Technical Activities Board (appointed by the Vice President for Technical Activities), and a member of the AdCom (appointed by the President of RAS).  The Vice President for Conference Activities, the President of RAS and the IEEE RAS Executive Director are ex-officio members. The Associate Vice President for Technical Programs will chair the CAB Technical Programs Committee. The Vice President for Conference Activities may appoint additional members to ensure that other RAS conferences are represented.



\subsection{Competitions Committee}
This committee shall report to the Vice President for Conferences Activities. This committee fosters and coordinates all competitions at IEEE RAS (co)sponsored events.  It evaluates requests for competitions and makes recommendations as to RAS endorsement, technical sponsorship, or financial support.  It also provides assistance to local chapters wishing to organize local or regional competitions. The Associate Vice President for Conference Competitions will chair the committee, and its membership typically includes representatives of events holding competitions, AdCom, the chair of the Education Committee, and a representative from the Student Activities Committee.



\subsection{Publication Ethics Committee}

This committee shall report to the Vice President for Publication Activities and Vice President for Conference Activities. The Committee shall be responsible for making recommendations about plagiarism issues in RAS sponsored publications and conferences.


\subsection{ICRA Steering Committee}

The ICRA Steering Committee is a subcommittee of the IEEE RAS Conference Activities Board (CAB). The ICRA Steering Committee Chair reports to the IEEE Robotics and Automation Society (RAS) Vice-President of Conference Activities or his/her designee. The ICRA Steering Committee oversees the long-term planning and success of current instances of the IEEE International Conference on Robotics and Automation (ICRA); plans for future editions in the series of ICRA conferences; evaluates how well each ICRA has achieved its objectives; and proposes and implements improvements to continuously meet these objectives. The ICRA Steering Committee facilitates the entire application and evaluation process for future ICRA teams and sites, including providing Adcom with detailed evaluation reports of the bids prior to final selection by AdCom. Details on the composition and responsibilities of the committee members are defined in the ICRA Steering Committee Charter.

\section{Conference Editorial Board}
\label{CEB}

\subsection{Purpose}

The ICRA Conference Editorial Board (CEB) is charged with reviewing papers for ICRA and maintaining year-to-year consistency in the quality of the reviewing process. The results of the review process are transmitted to the ICRA Program Chair for final decision on acceptance. 


\subsection{Composition}

The CEB consists of an Editor-in-Chief (EIC), several Editors (Eds) and Associate Editors (AEs). The number of Editors and Associate Editors is decided based on the number of submissions to be handled. The term of the Editor-in-Chief (EIC), Editors and all CEB members is three years, once renewable.The Editor-in-Chief (EIC) is appointed by the President of IEEE RAS upon recommendation by the Conference Technical Programs Committee (CAB-T). and endorsement by the Vice President for Conference Activities. The Editors will be appointed by the EIC, after approval by the Conference Technical Programs Committee (CAB-T). Each Associate Editor will be appointed by an Editor, after approval by the EIC.

\subsection{Operation}

The CEB will handle the paper review process for each ICRA through an appropriately chosen conference management software package. The CEB collects reviews and provides recommendations on acceptance, but does not make the final decision. The review results are transmitted by the EIC to the Program Chair of the corresponding ICRA, who then makes the final decision on acceptance. 

\subsection{Reports}

The EIC will make reports on the functioning of the CEB and of the progress and results of reviewing at each CAB-T meeting.


\section{Industrial Activities Board}
\label{IAB}
\subsection{Purpose}

The Industrial Activities Board shall be responsible for all industrial related matters and the growth of the industrial community within RAS, both by promoting the participation of industrial partners in current activities and thus creating more links with academics, and by proposing new actions beneficial to this community.

\subsection{Composition}
This Board will be composed of the Chair (VP Industrial Activities), the President, the Associate VPs Industrial Activities, and a minimum of four additional members appointed by the Vice President for Industrial Activities with approval of the President. At least one member of the board must be an elected member of the AdCom. The Standards Committee shall report to the Vice President for Industrial Activities.

\subsection{Actions}
The Industrial Activities Board shall meet at least two times a year to review the industrial activities of the Society.

\subsection{Standards Activities}
This committee shall report to the Industrial Activities Board. The Standing Committee for Standards Activities works to formally adopt and confirm best practices in robotics and automation as standards. The Standing Committee will work with interested members of RAS in supporting standards defining activities in both established, mature application areas and nascent, emerging technologies related to robotics and automation. The Standards Committee shall pursue the following objectives: promote common measures and definitions in robotics and automation; promote measurability and comparability of robotics and automation technology; promote integrability, portability, and reusability of robotics and automation technology.

\section{Media Services Board}
\label{MSB}

\subsection{Purpose}

The Media Services Board (previously called the Electronic Products and Services Board) shall be responsible for the creation and management of society website and electronic services.

\subsection{Composition}

This Board will be composed of the Chair (Vice President for Media Services), the President, the Associate Vice Presidents for Media Services, the Vice President for Financial Activities, the Vice-President for Conference Activities,  the Vice President for Technical Activities, the Vice-President for Member Activities, the Vice-President for Publication Activities, the Vice-President for Industrial Activities, the Treasurer, and between one and five additional members appointed by the Vice President Media Services with the approval of the President. At least one member of the board must be an elected member of the AdCom.  

\subsection{Actions}

This Board shall meet at least two times a year to review the electronic products and services of the Society.


\section{Standing Committees and Working Groups}
\label{StandingCommittees}

\subsection{Chairs}

Chairs of Standing Committees shall be appointed for two-year terms by the President-Elect prior to becoming President with the concurrence of the AdCom.

\subsection{Advisory Committee}

The Committee, on request, shall give advice to the Executive Committee and other major committees of the Society. A primary function shall be to advise the appropriate Boards on the latest significant developments in the Field of Interest of the Society and in related fields. The Committee shall consist of the five (5) immediate Past-Presidents, the chair being the Junior Past-President.

\subsection{Awards Committee}

This Committee shall be chaired or co-chaired by the Senior Past-President, unless said person is unavailable, in which case the President shall appoint a replacement. This committee shall be responsible for: 1) appointing an award nomination committee for each society and IEEE level (within RAS fields of interest) award, 2) appointing one or more award selection committee(s) to review society level award nominations and select the winner(s), 3) appointing a Fellow Evaluation Committee, 4) reviewing all new award proposals to ensure that  there is no conflict with existing awards and that an appropriate description suitable for submission to IEEE has been prepared, 5) submission of new/modified award proposals to IEEE, and 6) reviewing existing awards periodically to recommend whether or not they should be continued.  

The Award Nomination Committee(s) shall be responsible for securing a suitable number of nominations of deserving candidates for the award(s) for which they are responsible.  The Award Selection Committee(s) shall be responsible for reviewing the nomination packets of the candidates for the award(s) for which they are responsible and selecting the winner(s), if any. The Award Selection Committee(s) shall have no members who were members of the Nomination Committee of any award for which selection is to be made.

The Fellow Evaluation Committee shall be responsible for evaluating the qualifications of candidates for Fellow as requested by the IEEE Fellow Committee.  This committee shall be chaired or co-chaired by the Senior Past-President, unless said person is either unavailable or not a Fellow, in which case the President shall appoint a Fellow as chair. The Fellow Evaluation Committee will consist of the chair and at least 5 members. The committee shall have no members who either nominated or wrote a reference for any candidate to be considered.


\subsection{Constitution and Bylaws Committee}

This Committee shall be chaired by the Secretary and shall consist of its Chair, the Parliamentarian, the President, and additional members appointed by the Chair with approval of the President. The Committee will review the governing documents of the Society on an annual basis, and revise if needed.

\subsection{Executive Committee}

This Committee shall be chaired by the President and shall consist of its Chair, the President-Elect, all the Vice Presidents, the Treasurer and the Secretary. Between meetings of the AdCom, the Executive Committee shall be empowered to act for the Society except on matters which the AdCom has by resolution expressly reserved to itself.  The AdCom may by majority vote at any meeting override any act or decision of the Executive Committee. 

\subsection{Long Range Planning Committee}

This Committee shall be responsible to the AdCom for reviewing the trends of science and technology as they may concern the Society, the profession and the public, and for recommending changes in the objectives, organization, and operations of the Society as may be indicated by these trends. This Committee shall be chaired by the President-Elect and shall include the additional following members: the Junior Past-President, all the Vice Presidents, the Chair of the Student Activities Committee and additional members appointed by the Chair with approval of the President.

\subsection{Nominations Committee}

On or before January 31 of each year, the President shall appoint, with the prior approval of the Administrative Committee (AdCom), a Nominations Committee consisting of five Society members. The Junior Past-President will chair this committee, and at least one other member must be an AdCom member. In accordance with the procedures specified in \secref{Elections:NominationsCommittee} of the Bylaws, the Nominations Committee is charged with overseeing 1) nominations for the AdCom election, and 2) nominations for Society officer positions. The Nominations Committee is responsible for ensuring that all required information about nominees for AdCom and Society officer positions is communicated in a timely fashion to IEEE. The Nominations Committee is also responsible for submitting 1) nominations for Division Director to the IEEE Division Nominations Committee, and 2) a list of potential candidates to serve on various IEEE Committees and Boards to the IEEE Nominations and Appointments Committee.

\subsection{Education Activities Committee}

This committee oversees coordinating and fostering educational initiatives in the Society.  An important means toward this goal are specialized meetings, workshops, and dedicated conference sessions on the education of students. Society-sponsored books and summer schools/technical education programs are also under the responsibility of this Committee. This Committee provides oversight/direction of educational resources made available in electronic form. A concern of the society shall be to ensure that the quality of the educational activities sponsored by the Society is constantly checked through feedback from members, and by emphasizing their interactive participation.

\subsection{Committee to Aid Reporting on Misconduct Concerns (CARES)}

The Committee to Aid Reporting on Misconduct Concerns (CARES) is a committee that acts as a trusted listener and empathetic support for the RAS community members around misconduct concerns, including, but not limited to sexual harassment, discrimination, and bullying, at RAS conferences, workshops, meetings, and events. All IEEE RAS events abide by the IEEE Policy Against Discrimination and Harassment and the IEEE RAS Code of Conduct. If a violation of this policy occurs, IEEE RAS encourages reporting the incident as described in the RAS Code of Conduct. This committee shall operate in alignment with IEEE Bylaws and policies regarding ethical complaints.

\subsection{Additional Committees}

Additional committees may be formed when deemed necessary by the AdCom.   

\section{Liaison Representatives}
\subsection{Representation in the IEEE}

The President shall appoint, as required, members of the Society as Liaison Representatives to represent the Society on various Boards and Committees of the IEEE.

\subsection{Representation in Other Organizations}

The President shall appoint, as required, members of the Society as Liaison Representatives to represent the Society in dealings with other non-IEEE organizations.

\subsection{Terms of Office}

Liaison representatives shall be appointed by the President with the concurrence of the AdCom for one-year terms which may be renewed.

\section{Affiliate Membership}

Members in good standing of other professional organizations who satisfy the requirements stated in the Constitution, Article III, Section 2, are eligible for Affiliate Membership in the Society. Affiliate applications shall be reviewed and evaluated by the Society's Membership, Admissions and Retention Committee.

\section{Annual General Assembly}

The President shall call a general assembly of the Society membership yearly at a major conference.  Society officers and AdCom members shall be invited to this meeting.


\section{Absence of the President}

In the absence or incapacity of the President, the duties of the office shall be performed by the President-Elect or by a Vice President designated by the President.

\section{Society Executive Office}

Subject to compliance with all applicable IEEE Bylaws and Policies, the Society may create an Executive Office supported by IEEE staff.  The Society’s Executive Office functions to coordinate and carry-out the day-to-day operations, policies, and procedures concerning specific aspects of the Society’s business.  The Office may also maintain the corporate memory and provide ongoing and ad hoc management reports/documents.  In addition, the Society’s Executive Office serves as one of the Society’s primary points of contact for both members and IEEE staff. 

Subject to compliance with all applicable IEEE Bylaws and Policies, the Society may determine the budget for the Executive Office.  The staff is hired by the IEEE and all conditions of employment will be based upon IEEE Bylaws, staff policies and practices and all applicable laws and regulations.  Office organization, job descriptions, IEEE staff policies and employment practices are available from the IEEE Human Resources Department.  

The Executive Director is the most senior position on the IEEE paid staff that supports the Society and as such, he/she manages and develops, personally and through subordinate management staff, the paid IEEE staff members that support the Society’s operations and activities.  The Society Executive Director supports the Society president, officers and volunteer leadership to achieve the Society goals. This Society Executive Director directly reports through the Managing Director, Technical Activities, to the IEEE Executive Director.


\section{Amendments}

\subsection{Procedure}

Procedure for amending Bylaws is described in the Constitution, Article XI, Section 3. 





\end{document}